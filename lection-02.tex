\documentclass[aspectratio=169]{beamer}
\usepackage[utf8]{inputenc}
\usepackage[english,russian]{babel}
\usepackage{amssymb}
\usepackage{stmaryrd}
\usepackage{cmll}
\usepackage{xcolor}
\usepackage{proof}
\setbeamertemplate{navigation symbols}{}
\usepackage{tikz}
\usetikzlibrary{hobby,fit,backgrounds,calc,shapes.geometric,patterns}
%\usetheme{Warsaw}
\begin{document}

\newtheorem{axiom}{Аксиома}
\newtheorem{exmprus}{Пример}
\newtheorem{defrus}{Определение}
\newtheorem{lemmarus}{Лемма}
\newtheorem{thmrus}{Теорема}

\begin{frame}{}
\begin{center}\Large Теоремы об исчислении высказываний.\end{center}
\end{frame}

\begin{frame}{Напоминание: истинность}
\begin{itemize}
\item Если $\alpha$ истинна при любой оценке переменных, то $\alpha$ общезначима: $$\models\alpha$$
\item Если $\alpha$ истинна при любой оценке переменных, при которой истинны 
высказывания $\gamma_1, \dots, \gamma_n$, будем говорить, что $\alpha$ --- \emph{следствие} этих высказываний:
$$\gamma_1, \dots, \gamma_n \models \alpha$$\pause
\item Истинна при какой-нибудь оценке --- \emph{выполнима}.\pause
\item Не истинна ни при какой оценке --- \emph{невыполнима}.\pause
\item Не истинна при какой-нибудь оценке --- \emph{опровержима}.
\end{itemize}
\end{frame}

\begin{frame}{Выводимость из гипотез}

\begin{defrus}[доказательство формулы $\alpha$] 
--- такое доказательство (вывод) $\delta_1, \delta_2, \dots, \delta_n$,
что $\alpha\equiv\delta_n$.

Формула $\alpha$ доказуема (выводима), если существует её доказательство. Обозначение:
$$\vdash \alpha$$\end{defrus}\pause

\begin{defrus}[вывод формулы $\alpha$ из гипотез $\gamma_1,\dots,\gamma_k$]
--- такая последовательность
$\delta_1,\dots,\delta_n$, причём каждое $\delta_i$ либо:
\begin{itemize}
\item является аксиомой;
\item либо получается по правилу Modus Ponens из предыдущих;
\item либо является одной из гипотез: существует $t: \delta_i \equiv \gamma_t$.
\end{itemize}

Формула $\alpha$ выводима из гипотез $\gamma_1,\dots,\gamma_k$, если существует её вывод. Обозначение:
$$\gamma_1,\dots,\gamma_k\vdash\alpha$$\end{defrus}

\end{frame}

\begin{frame}{Корректность и полнота}
\begin{defrus}[корректность теории]
Теория корректна, если любое доказуемое в ней утверждение общезначимо.
То есть, $\vdash\alpha$ влечёт $\models\alpha$.
\end{defrus}

\begin{defrus}[полнота теории]
Теория семантически полна, если любое общезначимое в ней утверждение доказуемо.
То есть, $\models\alpha$ влечёт $\vdash\alpha$.
\end{defrus}
\end{frame}

\begin{frame}{Корректность исчисления высказываний}
\begin{thmrus}[корректность]
Если $\vdash\alpha$, то $\models\alpha$
\end{thmrus}

\begin{proof}
Индукция по длине вывода $n$.
\begin{itemize}
\item База, $n=1$ --- частный случай перехода (без правила Modus Ponens)
\item Переход. Пусть для любого доказательства длины $n$ формула $\delta_n$ общезначима.
Тогда рассмотрим обоснование $\delta_{n+1}$ и разберём случаи:
\begin{enumerate}
\item Аксиома --- убедиться, что все аксиомы общезначимы.
\item Modus Ponens $j$, $k$ --- убедиться, что если $\models\delta_j$ и 
$\models\delta_j\rightarrow\delta_{n+1}$, то $\models\delta_{n+1}$.
\end{enumerate}
\end{itemize}
\end{proof}
\end{frame}

\begin{frame}{Общезначимость схемы аксиом №9}
Общезначимость схемы аксиом --- истинность каждой аксиомы, задаваемой данной схемой, при любой оценке:
$$\llbracket(\alpha\rightarrow\beta)\rightarrow(\alpha\rightarrow\neg\beta)\rightarrow\neg\alpha\rrbracket
   = \textnormal{И}$$

Построим таблицу истинности формулы в зависимости от оценки $\alpha$ и $\beta$:
\vspace{0.3cm}

{\footnotesize
\begin{tabular}{cc|ccccc}
$\llbracket\alpha\rrbracket$ & $\llbracket\beta\rrbracket$ & $\llbracket\neg\alpha\rrbracket$ & 
    $\llbracket\alpha\rightarrow\beta\rrbracket$ & $\llbracket\alpha\rightarrow\neg\beta\rrbracket$ 
  & $\llbracket(\alpha\rightarrow\neg\beta)\rightarrow\neg\alpha\rrbracket$ & 
    $\llbracket(\alpha\rightarrow\beta)\rightarrow(\alpha\rightarrow\neg\beta)\rightarrow\neg\alpha\rrbracket$\\
\hline
  Л & Л & И & И & И & И & И\\
  Л & И & И & И & И & И & И\\
  И & Л & Л & Л & И & Л & И\\
  И & И & Л & И & Л & И & И
\end{tabular}}

\end{frame}

\begin{frame}{Общезначимость заключения правила Modus Ponens}
Пусть в выводе есть формулы $\delta_j$, $\delta_k \equiv \delta_j\rightarrow\delta_{n+1}$, $\delta_{n+1}$ (причём
$j < n+1$ и $k < n+1$).\vspace{0.3cm}\pause


Фиксируем какую-нибудь оценку. 
По индукционному предположению, $\delta_j$ и $\delta_j\rightarrow\delta_{n+1}$ общезначимы.
Поэтому при данной оценке $\llbracket\delta_j\rrbracket \equiv \textnormal{И}$ и
$\llbracket\delta_j\rightarrow\delta_{n+1}\rrbracket \equiv \textnormal{И}$.\vspace{0.3cm}\pause

Построим таблицу истинности для импликации:

\begin{center}\begin{tabular}{ccc}
$\llbracket\delta_j\rrbracket$ &$\llbracket\delta_{n+1}\rrbracket$ & $\llbracket\delta_j\rightarrow\delta_{n+1}\rrbracket$\\
\hline
Л & Л & И \\
Л & И & И \\
И & Л & Л \\
И & И & И
\end{tabular}\end{center}\pause

Из таблицы видно, что $\llbracket\delta_{n+1}\rrbracket = \textnormal{Л}$ только если 
$\llbracket\delta_j\rightarrow\delta_{n+1}\rrbracket = \textnormal{Л}$ или 
$\llbracket\delta_j\rrbracket = \textnormal{Л}$. Значит, это невозможно, и
$\llbracket\delta_{n+1}\rrbracket = \textnormal{И}$

\end{frame}

\begin{frame}{Контекст, метаязык}

Будем обозначать большими греческими буквами середины
алфавита, возможно с индексами, ($\Gamma$, $\Delta_1$, ...) списки формул.
Будем использовать, где удобно:

$$\Gamma \vdash \alpha$$\pause

Списки можно указывать через запятую:

$$\Gamma, \Delta, \zeta \vdash \alpha$$\pause
это означает то же, что и 
$$\gamma_1,\gamma_2,\dots,\gamma_n,\delta_1,\delta_2,\dots,\delta_m,\zeta\vdash\alpha$$
если 
$$\Gamma := \{\gamma_1,\gamma_2,\dots,\gamma_n\},\quad \Delta := \{\delta_1,\delta_2,\dots,\delta_m\}$$

\end{frame}

\begin{frame}{Теорема о дедукции}

\begin{theorem}[О дедукции, Жак Эрбран, 1930]
$\Gamma,\alpha\vdash\beta$ выполнено тогда и только тогда, когда выполнено $\Gamma\vdash\alpha\rightarrow\beta$
\end{theorem}\pause

\vspace{0.5cm}
Доказательство <<в две стороны>>, сперва <<справа налево>>.
Пусть $\Gamma\vdash\alpha\rightarrow\beta$, покажем $\Gamma,\alpha\vdash\beta$\pause\vspace{0.5cm}

То есть по условию существует вывод: $$\delta_1, \delta_2, \dots, \delta_{n-1}, \alpha\rightarrow\beta$$\pause

Тогда следующая последовательность --- тоже вывод: 
$$\delta_1, \delta_2, \dots, \delta_{n-1}, \alpha\rightarrow\beta, \alpha, \beta$$

%Рассмотрим формулу $(A \rightarrow B) \rightarrow (\neg B)$

\end{frame}

\begin{frame}{Доказательство: $\Gamma\vdash\alpha\rightarrow\beta$ влечёт $\Gamma,\alpha\vdash\beta$}
\begin{tabular}{lll}
№ п/п & формула & пояснение\\
\hline
$(1)$ & $\delta_1$ & в соответствии с исходным доказательством\\
    & $\dots$ \\
$(n-1)$ & $\delta_{n-1}$ & в соответствии с исходным доказательством\\
$(n)$ & $\alpha\rightarrow\beta $ & в соответствии с исходным доказательством\\
$(n+1)$ & $\alpha$ & гипотеза\\
$(n+2)$ & $\beta$ & Modus Ponens n$+1$, $n$
\end{tabular}\pause\vspace{1cm}

Вывод $\Gamma,\alpha\vdash\beta$ предоставлен, первая часть теоремы доказана.
\end{frame}

\begin{frame}{Доказательство: $\Gamma,\alpha\vdash\beta$ влечёт $\Gamma\vdash\alpha\rightarrow\beta$}

Пусть даны формулы вывода $$\delta_1,\delta_2,\dots,\delta_{n-1},\beta$$

Аналогично предыдущему пункту, перестроим вывод.\pause

Построим <<черновик>> вывода, приписав $\alpha$ слева к каждой формуле:
$$\alpha\rightarrow\delta_1,\alpha\rightarrow\delta_2,\dots,\alpha\rightarrow\delta_{n-1},\alpha\rightarrow\beta$$\pause
Данная последовательность формул не обязательно вывод: $\Gamma:=\varnothing$, $\alpha := A$
$$\delta_1 := A\rightarrow B\rightarrow A$$\pause
припишем $A$ слева --- вывод не получим:
$$\alpha\rightarrow\delta_1 \equiv A \rightarrow (A\rightarrow B\rightarrow A)$$
\end{frame}

\begin{frame}[fragile]{Последовательности, странная нумерация}

\begin{defrus}[конечная последовательность]
Функция $\delta: 1\dots n \rightarrow \mathcal{F}$
\end{defrus}

\begin{defrus}[конечная последовательность, индексированная дробными числами]
Функция $\zeta: I \rightarrow \mathcal{F}$, где $I \subset \mathbb{Q}$ и $|I| \in \mathbb{N}$
\end{defrus}

\begin{exmprus}[странный мотивационный пример: язык Фокал]
%\begin{exmrus}
\begin{tabular}{cp{1cm}|p{1cm}c}Программа &&& Вывод \\\hline
\begin{minipage}{0.4\textwidth}
\begin{verbatim}

10.1    t n,!
10.15   s n = n+1
10.17   i (n-3) 10.1,11.0,11.0
11.0    t "That's all"
\end{verbatim}
\end{minipage}
&&&
%{\color{gray}Что программа печатает:}
\begin{minipage}{0.3\textwidth}
\begin{verbatim}

=     0.0000
=     1.0000
=     2.0000
That's all
\end{verbatim}
\end{minipage}
\end{tabular}
\end{exmprus}
\end{frame}


\begin{frame}{Доказательство: $\Gamma,\alpha\vdash\beta$ влечёт $\Gamma\vdash\alpha\rightarrow\beta$}

\begin{proof} (индукция по длине вывода). Если $\delta_1, \dots, \delta_n$ --- вывод
$\Gamma,\alpha\vdash\beta$, то найдётся вывод $\zeta_k$ для $\Gamma\vdash\alpha\rightarrow\beta$,
причём $\zeta_1 \equiv \alpha\rightarrow\delta_1, \dots, \zeta_n \equiv \alpha\rightarrow\delta_n$.

\begin{itemize}
\item База $(n=1)$: частный случай перехода (без M.P.).

\item Переход. Пусть $\delta_1, \dots, \delta_{n+1}$ --- исходный вывод. И пусть (по индукционному предположению)
уже по начальному фрагменту $\delta_1, \dots, \delta_n$ построен вывод $\zeta_k$ утверждения 
$\Gamma\vdash\alpha\rightarrow\delta_n$. 

Но $\delta_{n+1}$ как-то был обоснован --- разберём случаи:
\begin{enumerate}
\item $\delta_{n+1}$ --- аксиома или $\delta_{n+1} \in \Gamma$ %(выполнено без доказательства в новом выводе)\pause
\item $\delta_{n+1}\equiv\alpha$\pause
\item $\delta_{n+1}$ --- Modus Ponens из $\delta_j$ и 
$\delta_k \equiv \delta_j\rightarrow\delta_{n+1}$.
\end{enumerate}

В каждом из случаев можно дополнить черновик до полноценного вывода.
\end{itemize}\end{proof}

\end{frame}

\begin{frame}{Доказательство: $\Gamma,\alpha\vdash\beta$ влечёт $\Gamma\vdash\alpha\rightarrow\beta$, случай аксиомы}
\begin{tabular}{lll}
№ п/п & новый вывод & пояснение \\
\hline
& \dots\\
$(1)$ & $\alpha\rightarrow\delta_1$ \\
& \dots\\
$(2)$ & $\alpha\rightarrow\delta_2$ \\
    & \dots \\
$(n)$ & $\alpha\rightarrow\delta_n$ \\
 & \color{cyan}$\alpha\rightarrow\delta_{n+1}$ & \color{cyan}$\delta_{n+1}$ --- аксиома, либо $\delta_{n+1} \in \Gamma$\\
\end{tabular}
\end{frame}

\begin{frame}{Доказательство: $\Gamma,\alpha\vdash\beta$ влечёт $\Gamma\vdash\alpha\rightarrow\beta$, случай аксиомы}
\begin{tabular}{lll}
№ п/п & новый вывод & пояснение \\
\hline
& \dots\\
$(1)$ & $\alpha\rightarrow\delta_1$ \\
& \dots\\
$(2)$ & $\alpha\rightarrow\delta_2$ \\
    & \dots \\
\color{cyan}$(n+0.3)$ & \color{cyan}$\delta_{n+1}\rightarrow\alpha\rightarrow\delta_{n+1}$ & \color{cyan}схема аксиом 1\\
\color{cyan}$(n+0.6)$ & \color{cyan}$\delta_{n+1}$ & \color{cyan}аксиома, либо $\delta_{n+1} \in \Gamma$\\
$(n+1)$ & $\alpha\rightarrow\delta_{n+1}$ & M.P. $n+0.6$, $n+0.3$\\
\end{tabular}
\end{frame}

\begin{frame}{Доказательство: $\Gamma,\alpha\vdash\beta$ влечёт $\Gamma\vdash\alpha\rightarrow\beta$, случай $\delta_i\equiv\alpha$}
\begin{tabular}{lll}
№ п/п & новый вывод & пояснение \\
\hline
    & \dots \\
$(1)$ & $\alpha\rightarrow\delta_1$ \\
    & \dots \\
$(2)$ & $\alpha\rightarrow\delta_2$ \\
    & \dots \\
\color{cyan}$(n+0.2)$ & \color{cyan}$\alpha \rightarrow (\alpha \rightarrow \alpha)$ & \color{cyan} Сх. акс. 1\\
\color{cyan}$(n+0.4)$ & \color{cyan}$(\alpha \rightarrow (\alpha \rightarrow \alpha)) \rightarrow 
  (\alpha \rightarrow (\alpha \rightarrow \alpha) \rightarrow \alpha) \rightarrow
  (\alpha \rightarrow \alpha)$& \color{cyan}Сх. акс. 2\\
\color{cyan}$(n+0.6)$ & \color{cyan}$(\alpha \rightarrow (\alpha \rightarrow \alpha) \rightarrow \alpha) \rightarrow
  (\alpha \rightarrow \alpha)$ &\color{cyan}M.P. $n+0.2$, $n+0.4$\\
\color{cyan}$(n+0.8)$ & \color{cyan}$\alpha \rightarrow (\alpha \rightarrow \alpha) \rightarrow \alpha$ & 
    \color{cyan}Сх. акс. 1\\
$(n+1)$ & $\alpha \rightarrow \alpha$ & M.P. $n+0.8$, $n+0.6$\\
\end{tabular}
\end{frame}

\begin{frame}{Доказательство: $\Gamma,\alpha\vdash\beta$ влечёт $\Gamma\vdash\alpha\rightarrow\beta$, случай Modus Ponens}
\begin{tabular}{lll}
№ п/п & новый вывод & пояснение \\
\hline
    & \dots \\
$(1)$ & $\alpha\rightarrow\delta_1$ \\
    & \dots \\
$(2)$ & $\alpha\rightarrow\delta_2$ \\
    & \dots \\
$(j)$ & $\alpha\rightarrow\delta_j$ \\
    & \dots \\
$(k)$ & $\alpha\rightarrow\delta_j\rightarrow\delta_{n+1}$ \\
    & \dots \\
\color{cyan}$(n+0.3)$ & \color{cyan}$(\alpha\rightarrow\delta_j)
    \rightarrow(\alpha\rightarrow\delta_j\rightarrow\delta_{n+1})\rightarrow(\alpha\rightarrow\delta_{n+1})$ & \color{cyan}Сх. акс. 2\\
\color{cyan}$(n+0.6)$ & \color{cyan}$(\alpha\rightarrow\delta_j
    \rightarrow\delta_{n+1})\rightarrow(\alpha\rightarrow\delta_{n+1})$ & \color{cyan}M.P. $j$, $n+0.3$\\
$(n+1)$ & $\alpha\rightarrow\delta_{n+1}$ & M.P. $k$, $n+0.6$\\
\end{tabular}
\end{frame}

\begin{frame}{Некоторые полезные правила}

\begin{lemmarus}[Правило контрапозиции]Каковы бы ни были формулы $\alpha$ и $\beta$, справедливо, что 
$\vdash (\alpha \rightarrow \beta) \rightarrow (\neg\beta \rightarrow \neg\alpha)$.
\end{lemmarus}\pause

\begin{lemmarus}[правило исключённого третьего]Какова бы ни была формула $\alpha$, $\vdash\alpha\vee\neg\alpha$.
\end{lemmarus}

\begin{lemmarus}[об исключении допущения]
Пусть справедливо $\Gamma, \rho \vdash \alpha$ и $\Gamma, \neg \rho \vdash \alpha$.
Тогда также справедливо $\Gamma \vdash \alpha$.
\end{lemmarus}

\begin{proof}Доказывается с использованием лемм, указанных выше.\end{proof}

\end{frame}

\begin{frame}{Теорема о полноте исчисления высказываний}
\begin{thmrus}Если $\models\alpha$, то $\vdash\alpha$.
\end{thmrus}
\end{frame}

\begin{frame}{Специальное обозначение}
\begin{defrus}[условное отрицание]
Зададим некоторую оценку переменных, такую, что $\llbracket\alpha\rrbracket = x$. 

Тогда \emph{условным отрицанием} формулы $\alpha$ назовём следующую формулу $\llparenthesis\alpha\rrparenthesis$:
$$\llparenthesis\alpha\rrparenthesis = \left\{\begin{array}{ll}\alpha, & x = \textnormal{И}\\
       \neg\alpha, & x = \textnormal{Л}\end{array}\right.$$

\end{defrus}

Аналогично записи для оценок, будем указывать оценку переменных, если это потребуется / будет неочевидно из контекста:
$$\llparenthesis \neg X \rrparenthesis^{X:=\textnormal{Л}} = \neg X\quad\quad\quad\llparenthesis \neg X \rrparenthesis^{X:=\textnormal{И}} = \neg\neg X$$

Также, если $\Gamma := \gamma_1, \gamma_2, \dots, \gamma_n$, то за $\llparenthesis \Gamma \rrparenthesis$ 
обозначим $\llparenthesis \gamma_1 \rrparenthesis, \llparenthesis \gamma_2 \rrparenthesis, \dots \llparenthesis \gamma_n \rrparenthesis$.
\end{frame}

\begin{frame}{Таблицы истинности и высказывания}

Рассмотрим связку <<импликация>> и её таблицу истинности:

\begin{center}\begin{tabular}{cccc}
$\llbracket A\rrbracket$ & $\llbracket B\rrbracket$ & $\llbracket A\rightarrow B\rrbracket$ & формула\\\hline
Л & Л & И & $\neg A, \neg B \vdash A \rightarrow B$\\
Л & И & И & $\neg A, B \vdash A \rightarrow B$\\
И & Л & Л & $A, \neg B \vdash \neg (A \rightarrow B)$\\
И & И & И & $A, B \vdash A \rightarrow B$
\end{tabular}\end{center}\pause

Заметим, что с помощью условного отрицания данную таблицу можно записать в одну строку:

$$\llparenthesis A \rrparenthesis, \llparenthesis B \rrparenthesis \vdash \llparenthesis A \rightarrow B \rrparenthesis $$

\end{frame}


\begin{frame}{Полнота исчисления высказываний}

\begin{thmrus}[О полноте исчисления высказываний]
Если $\models\alpha$, то $\vdash\alpha$
\end{thmrus}\pause

\begin{enumerate}
\item Построим таблицы истинности для каждой связки $(\star)$ и докажем в них каждую строку:
$$ \llparenthesis\varphi\rrparenthesis, \llparenthesis\psi\rrparenthesis \vdash \llparenthesis\varphi\star\psi\rrparenthesis$$\pause
\vspace{-0.5cm}
\item Построим таблицу истинности для $\alpha$ и докажем в ней каждую строку:
$$\llparenthesis \Xi \rrparenthesis \vdash \llparenthesis \alpha \rrparenthesis$$\pause
\vspace{-0.5cm}
\item Если формула общезначима, то в ней все строки будут иметь вид $\llparenthesis \Xi \rrparenthesis \vdash\alpha$,
потому от гипотез мы сможем избавиться и получить требуемое $\vdash\alpha$.
\end{enumerate}

\end{frame}

\begin{frame}{Шаг 1. Лемма о связках}

Запись

$$\llparenthesis\varphi\rrparenthesis, \llparenthesis\psi\rrparenthesis \vdash \llparenthesis\varphi\star\psi\rrparenthesis$$

сводится к 14 утверждениям:

\begin{center}\begin{tabular}{rclp{1cm}rcl}
$\neg\varphi, \neg\psi$&$ \vdash $&$\neg (\varphi \with \psi)$& & $\neg\varphi, \neg\psi$&$ \vdash $&$     (\varphi \rightarrow  \psi)$ \\
$\neg\varphi,     \psi$&$ \vdash $&$\neg (\varphi \with \psi)$& &$\neg\varphi,     \psi$&$ \vdash $&$     (\varphi \rightarrow  \psi)$ \\
$    \varphi, \neg\psi$&$ \vdash $&$\neg (\varphi \with \psi)$& &$ \varphi, \neg\psi$&$ \vdash $&$\neg (\varphi \rightarrow  \psi)$ \\
$    \varphi,     \psi$&$ \vdash $&$     (\varphi \with \psi)$& &$    \varphi,     \psi$&$ \vdash $&$     (\varphi \rightarrow  \psi)$ \\
$\neg\varphi, \neg\psi$&$ \vdash $&$\neg (\varphi \vee  \psi)$& &$    \varphi          $&$ \vdash $&$     \neg\neg\varphi$ \\
$\neg\varphi,     \psi$&$ \vdash $&$     (\varphi \vee  \psi)$& &$\neg\varphi          $&$ \vdash $&$         \neg\varphi$\\
$    \varphi, \neg\psi$&$ \vdash $&$     (\varphi \vee  \psi)$ \\
$    \varphi,     \psi$&$ \vdash $&$     (\varphi \vee  \psi)$
\end{tabular}\end{center}
\end{frame}

\begin{frame}{Шаг 2. Обобщение на любую формулу}

\begin{lemmarus}[Условное отрицание формул]
Пусть пропозициональные переменные $\Xi := \{X_1, \dots, X_n\}$ ---
все переменные, которые используются в формуле $\alpha$. И пусть
задана некоторая оценка переменных.

Тогда, $\llparenthesis \Xi \rrparenthesis \vdash\llparenthesis\alpha\rrparenthesis$
\end{lemmarus}\pause

\begin{proof}Индукция по длине формулы $\alpha$.
\begin{itemize}
\item База: формула $\alpha$ --- атомарная, т.е. $\alpha \equiv X_i$. Тогда при любом $\Xi$ выполнено 
$\llparenthesis\Xi\rrparenthesis^{X_i := \text{И}} \vdash X_i$ и $\llparenthesis\Xi\rrparenthesis^{X_i := \text{Л}} \vdash \neg X_i$.
\item Переход: $\alpha \equiv \varphi\star\psi$, причём $\llparenthesis\Xi\rrparenthesis\vdash\llparenthesis\varphi\rrparenthesis$
и $\llparenthesis\Xi\rrparenthesis\vdash\llparenthesis\psi\rrparenthesis$\pause

Тогда построим вывод: 

\begin{tabular}{lll}
$(1)\dots(n)$ & $\llparenthesis\varphi\rrparenthesis$ & индукционное предположение\\
$(n+1)\dots(k)$ & $\llparenthesis\psi\rrparenthesis$ & индукционное предположение\\
$(k+1)\dots(l)$ & $\llparenthesis\varphi\star\psi\rrparenthesis$ & 
  лемма о связках: $\llparenthesis\varphi\rrparenthesis$ и $\llparenthesis\psi\rrparenthesis$ доказаны выше,\\
  & & значит, их можно использовать как гипотезы
\end{tabular}
\end{itemize}
\end{proof}

\end{frame}

\begin{frame}{Шаг 3. Избавляемся от гипотез}

\begin{lemmarus}Пусть при всех оценках переменных
$\llparenthesis\Xi\rrparenthesis \vdash \alpha$, тогда
$\vdash\alpha$.
\end{lemmarus}\pause

\begin{proof}
Индукция по количеству переменных $n$.

\begin{itemize}
\item База: $n=0$. Тогда $\vdash\alpha$ есть из условия.\pause
\item Переход: пусть $\llparenthesis X_1, X_2,  \dots X_{n+1} \rrparenthesis \vdash \alpha$.
Рассмотрим $2^n$ пар выводов:
$$\infer
  {\llparenthesis X_1, X_2, \dots X_n \rrparenthesis \vdash \alpha}
  {\llparenthesis X_1, X_2, \dots X_n\rrparenthesis,\neg X_{n+1} \vdash \alpha\quad\quad 
   \llparenthesis X_1, X_2, \dots X_n\rrparenthesis,X_{n+1} \vdash \alpha
}$$
\end{itemize}
При этом, $\llparenthesis X_1, X_2, \dots X_n \rrparenthesis  \vdash \alpha$ при всех оценках
переменных $X_1, \dots X_n$. Значит, $\vdash\alpha$ по индукционному предположению.
\end{proof}

\end{frame}

\begin{frame}{Заключительные замечания}
Теорема о полноте --- конструктивна. Получающийся вывод --- экспоненциальный по длине.

Несложно по изложенному доказательству разработать программу, строящую вывод.

Вывод для формулы с 3 переменными --- порядка 3 тысяч строк.

\end{frame}

\end{document}
